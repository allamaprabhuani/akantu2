
\subsection{Cohesive laws}

\subsubsection{Snozzi-Molinari law}

\begin{figure}
  \centering
  \subfloat[Linear]{\includegraphics[width=0.4\textwidth]{figures/linear_cohesive_law}}
  \qquad
  \subfloat[Bilinear]{\includegraphics[width=0.4\textwidth]{figures/bilinear_cohesive_law}}
  \caption{Irreversible cohesive laws for explicit simulations.}
  \label{fig:smm:coh:linear_cohesive_law}
\end{figure}

The Snozzi-Molinari~\cite{snozzi_cohesive_2013} linear irreversible
cohesive law has been implemented in \akantu (see
figure~\ref{fig:smm:coh:linear_cohesive_law}). It is an extension to
the Camacho-Ortiz~\cite{camacho_computational_1996} cohesive law in
order to make dissipated fracture energy path-dependent. The concept
of free potential energy is dropped and a new independent parameter
$\kappa$ is introduced:
\begin{equation}
  \kappa = \frac{G_\mathrm{c, II}}{G_\mathrm{c, I}}
\end{equation}
where $G_\mathrm{c, I}$ and $G_\mathrm{c, II}$ are respectively the
necessary works of separation per unit area to open completely a
cohesive zone under mode I and mode II. Their model yields to the
following equation for cohesive tractions $\vec{T}$ in case of crack
opening ${\delta}$:
\begin{equation}
  \label{eq:smm:coh:tractions}
  \vec{T} = \left( \frac{\beta^2}{\kappa} \Delta_\mathrm{t} \vec{t} +
    \Delta_\mathrm{n} \vec{n} \right)
  \frac{\sigma_\mathrm{c}}{\delta}
  \left( 1- \frac{\delta}{\delta_\mathrm{c}} \right)
\end{equation}
where $\sigma_\mathrm{c}$ is the material strength along the fracture,
$\delta_\mathrm{c}$ the critical effective displacement after which
cohesive tractions are zero (complete decohesion), $\Delta_\mathrm{t}$
and $\Delta_\mathrm{n}$ are respectively the tangential and normal
components of the opening displacement vector $\vec{\Delta}$. $\beta$
specifies the importance of tangential to the normal displacement. The
effective opening displacement is:
\begin{equation}
  \delta = \sqrt{\frac{\beta^2}{\kappa^2} \Delta_\mathrm{t}^2 +
    \Delta_\mathrm{n}^2}
\end{equation}
In case of unloading or reloading $\delta < \delta_\mathrm{max}$,
tractions are calculated as:
\begin{align}
  T_\mathrm{n} &= \Delta_\mathrm{n}\,
  \frac{\sigma_\mathrm{c}}{\delta_\mathrm{max}}
  \left( 1- \frac{\delta_\mathrm{max}}{\delta_\mathrm{c}} \right) \\
  T_\mathrm{t} &= \frac{\beta^2}{\kappa}\, \Delta_\mathrm{t}\,
  \frac{\sigma_\mathrm{c}}{\delta_\mathrm{max}}
  \left( 1- \frac{\delta_\mathrm{max}}{\delta_\mathrm{c}} \right)
\end{align}
so that they vary linearly between the origin and the maximum attained
tractions. As shown in figure~\ref{fig:smm:coh:linear_cohesive_law},
in this law dissipated and reversible energies are:
\begin{align}
  E_\mathrm{diss} &= \frac{1}{2} \sigma_\mathrm{c}\, \delta_\mathrm{max}\\[1ex]
  E_\mathrm{rev} &= \frac{1}{2} T\, \delta
\end{align}
Moreover a damage parameter $D$ can be defined as:
\begin{equation}
  D = \min \left(
    \frac{\delta_\mathrm{max}}{\delta_\mathrm{c}},1 \right)
\end{equation} which varies from 0 (undamaged condition) and 1 (fully
damaged condition). This variable can only increase because damage is
an irreversible process. A simple penalty contact model has been incorporated
in the cohesive law so that normal tractions can be returned in
case of compression:
\begin{equation}
  T_\mathrm{n} = \alpha \Delta_\mathrm{n} \quad\text{if
    $\Delta_\mathrm{n} < 0$}
\end{equation}
where $\alpha$ is a stiffness parameter that by default is zero. The
relative contact energy is equivalent to reversible energy but in
compression.

The material name of the linear decreasing cohesive law  is
\code{material\_cohesive\_linear} and its parameters with their
relative default values are:
\begin{itemize}
\item \code{sigma\_c}: 0
\item \code{beta}: 0
\item \code{G\_cI}: 0
\item \code{G\_cII}: 0
\item \code{kappa}: 1
\item \code{penalty}: 0
\end{itemize}
A random number generator can be use to assign a random
$\sigma_\mathrm{c}$ to each facet following a given distribution. To
the base value of $\sigma_\mathrm{c}$ a random quantity can be added:
\begin{cpp}
  sigma_c = $base$ uniform [$minimum$, $maximum$]
  sigma_c = $base$ weibull [$lambda$, $m$]
\end{cpp}
All parameters are real numbers. For the uniform distribution minimum
and maximum values have to be specified. Weibull distribution is
characterized by the following cumulative distribution function:
\begin{equation}
    F(\sigma_\mathrm{c}) = 1- e^{-\left(
      \frac{\sigma_c-\sigma_\mathrm{c, min}}{\lambda} \right)^m}
\end{equation}
which depends on $\sigma_\mathrm{c, min}$, which is the minimum value,
and $m$ and $\lambda$, which are the shape parameter and the scale
parameter.

Bilinear constitutive law works exactly in the same way of the linear
one, except for the additional parameter \code{delta\_0} that by
default is zero. Two examples for the extrinsic and intrinsic cohesive
elements and also an example to assign different porperties to
intergranular and transgranular cohesive elements can be found in
\code{\examplesdir/cohesive\_element/} folder.

\subsubsection{Exponential cohesive law}

Ortiz and Pandolfi proposed this cohesive law in 1999~\cite{ortiz1999}.  The
traction-opening equation for this law is as following:

\begin{equation}
  \label{eq:exponential_law}
  t = e \sigma_c \frac{\delta}{\delta_c}e^{-\delta/ \delta_c}
\end{equation}


This equation is plotted in figure (\ref{fig:smm:CL:ECL}).

 \begin{figure}[!htb]
    \begin{center}
      \includegraphics[width=0.6\textwidth,keepaspectratio=true]{figures/cohesive_exponential.pdf}
      \caption{Exponential cohesive law}
      \label{fig:smm:CL:ECL}
    \end{center}
  \end{figure}



\subsubsection{Static cohesive element}

For  the  static  analysis   of  the  structures  containing  cohesive
elements, the stiffness of the cohesive elements should also be added to the
total      stiffness       of      the      structure.      Therefore,
the  function  \code{assembleStiffnessMatrix}  is written  in  the
\code{MaterialCohesive}.

Considering a typical cohesive element as explained in figure~\ref{fig:smm:coh:cohesive2d}, the opening
displacement along the mid-surface can be written as:

\begin{equation}
  \label{eq:opening}
  \delta(\xi) = [[\mat{u}]] \mat{N}(\xi) =
\begin{bmatrix}
u_3-u_0 & u_4-u_1 & u_5-u_2\\
v_3-v_0 & v_4-v_1 & v_5-v_2
\end{bmatrix}
\begin{bmatrix}
N_0(\xi) \\ N_1(\xi) \\ N_2(\xi)
\end{bmatrix} =
\mat{N A U}
\end{equation}

 The \mat{U} , \mat{A} and \mat{N} are as following:

\begin{equation}
  \mat{U} = \left [
\begin{array}{c c c c c c c c c c c c}
u_0 & v_0 & u_1 & v_1 & u_2 & v_2 & u_3 & v_3 & u_4 & v_4 & u_5 & v_5
\end{array}\right ]
\end{equation}


\begin{equation}
  \mat{A} = \left [\begin{array}{c c c c c c c c c c c c}
1 & 0 & 0 & 0& 0 & 0 & -1& 0 & 0 &0 &0 &0\\
0 &1& 0&0 &0 &0 &0 & -1& 0& 0 & 0 &0\\
0 &0& 1&0 &0 &0 &0 & 0& -1& 0 & 0 &0\\
0 &0& 0&1 &0 &0 &0 & 0& 0& -1 & 0 &0\\
0 &0& 0&0 &1 &0 &0 & 0& 0& 0 & -1 &0\\
0 &0& 0&0 &0 &1 &0 & 0& 0& 0 & 0 &-1
\end{array} \right ]
\end{equation}


 \begin{equation}
 \mat{N} = \begin{bmatrix}
N_0(\xi) & 0 & N_1(\xi) &0 & N_2(\xi) & 0\\
0 & N_0(\xi)& 0 &N_1(\xi)& 0 & N_2(\xi)
\end{bmatrix}
\end{equation}

The consistent stiffness matrix for the element is obtained as

\begin{equation}
  \label{eq:cohesive_stiffness}
  \mat{K}    =    \delta    \mat{U}^T    \int_{\Gamma_c}    {\mat{P}^t
    \frac{\partial{\mat{t}}} {\partial{\delta}} \mat{P} d
    \Gamma \Delta \mat{U}}
\end{equation}

In which the  tangent matrix is calculated based  on the equation \ref
{eq:tangent_cohesive} after
performing necessary derivation:

\begin{equation}
  \label{eq:tangent_cohesive}
   \frac{\partial{\mat{t}}} {\partial{\delta}} = \hat{\mat{t}} \otimes
   \frac                       {\partial{(t/\delta)}}{\partial{\delta}}
   \frac{\hat{\mat{t}}}{\delta}+ \frac{t}{\delta}  [ \beta^2 \mat{I} +
   (1-\beta^2) (\mat{n} \otimes \mat{n})]
\end{equation}

 In which $\frac{\partial{(t/\delta)}}{\partial{\delta}}$ in the first
 term is

\begin{equation}
 \frac{\partial{(t/ \delta)}}{\partial{\delta}} = \left\{\begin{array} {l l}
-e  \frac{\sigma_c}{\delta_c^2  }e^{-\delta  /  \delta_c} &  \quad  if
\delta \geq \delta_{max}\\
 0 & \quad if \delta < \delta_{max}, \delta_n > 0
\end{array} \right.
\end{equation}

This      tangent      matrix      is     implemented      in      the
\code{computeTangentStiffness} function for
the exponential law.



%%% Local Variables:
%%% mode: latex
%%% TeX-master: "manual"
%%% End:
