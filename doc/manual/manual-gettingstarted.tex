\section{Getting started}
\subsection{Downloading the code}
SVN URL to get \akantu :
\begin{command}
  > svn co svn+ssh://username@intranet-lsms.epfl.ch/space/repositories/SimulPack/akantu/trunk akantu
\end{command}

\subsection{Compiling Akantu}
\begin{command}
  > cd akantu
  > mkdir build
  > cd build
  > ccmake <path to akantu sources>
\end{command}

Set the options that you need

\begin{command}
  > make
  > make install
\end{command}

\subsection{Creating and loading a mesh\label{sect:common:mesh}}

\akantu supports meshes generate with Gmsh~\cite{gmsh}, a free
software that is available in the
internet\footnote{http://geuz.org/gmsh/}. Its documentation can be
found in the website, so in this manual no directions for Gmsh usage
will be provided. Gmsh outputs meshes in .msh format, that can be read
by \akantu. In order to import a mesh, first it is necessary to create
a Mesh object through the following commands:
\begin{cpp}
  UInt spatial_dimension = 2;
  Mesh mesh(spatial_dimension);
\end{cpp}
The only parameter that has to be specified by the user is the spatial
dimension of the problem. Now it is possible to read a .msh file
with a MeshIOMSH object that takes care of mesh input and output. This
step is carried out by:
\begin{cpp}
  MeshIOMSH mesh_io;
  mesh_io.read("square.msh", mesh);
\end{cpp}
where first the MeshIOMSH object is created and then it is used to
read the .msh file. The mesh file name and Mesh object where to store
it must be specified by the user.
