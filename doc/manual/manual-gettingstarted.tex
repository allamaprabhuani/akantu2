\chapter{Getting started}
\section{Downloading the code}

The \akantu source code must be requested to the LSMS by
filling the form accessible at the URL \url{http://lsms.epfl.ch/akantu}.
There, you will be asked to accept the LGPL license terms.

\section{Compiling Akantu}

Akantu is a \code{cmake} project, so to configure it you can either follow the usual way:
\begin{command}
  > cd akantu
  > mkdir build
  > cd build
  > ccmake <path to akantu sources>
  [ Set the options that you need ]
  > make
  > make install
\end{command}

Or you can use a tool that we added to help you. You can just use the
given \code{Makefile} that handles the \code{cmake} configuration

\begin{command}
  > cd akantu
  > make config
  > make
  > make install
\end{command}

\section{Writing a \code{main} function\label{sect:common:main}}

First of all, \akantu needs to be initialized.  There is a memory
management included in the core library which allows a correct
allocation and de-allocation of vectors, structures and/or
objects. Moreover, in parallel computations the initialization
procedure can perform the communication setup. This is achieved by
utilizing a pair of functions (\code{initialize} and \code{finalize})
that are used as follows:
\begin{cpp}
int main(int argc, char *argv[])
{
  akantu::initialize(argc, argv);

  // your code
  ...

  akantu::finalize();
}
\end{cpp}
The \code{initialize} function takes the program parameters which
can be interpreted by \akantu in due form.

\section{Creating and loading a mesh\label{sect:common:mesh}}

\akantu supports meshes generated with Gmsh~\cite{gmsh}, a free
software available at \url{http://geuz.org/gmsh/} where a detailed
documentation can be found. Consequently this manual will not provide
Gmsh usage directions. Gmsh outputs meshes in \textit{.msh} format, that can be read
by \akantu. In order to import a mesh, it is necessary to create
a \code{Mesh} object through the following function calls:
\begin{cpp}
  UInt spatial_dimension = 2;
  Mesh mesh(spatial_dimension);
\end{cpp}
The only parameter that has to be specified by the user is the spatial
dimension of the problem. Now it is possible to read a \textit{.msh} file with
a \textit{MeshIOMSH} object that takes care of loading a mesh to memory.
This step is carried out by:
\begin{cpp}
  MeshIOMSH mesh_io;
  mesh_io.read("square.msh", mesh);
\end{cpp}
where the \textit{MeshIOMSH} object is first created before being
used to read the \text{.msh} file. The mesh file name as well as the \textit{Mesh}
object must be specified by the user.

The \textit{MeshIOMSH} object can also write mesh files. This feature
is useful to save a mesh that has been modified during a
simulation. The \code{write} method takes care of it:
\begin{cpp}
  mesh_io.write("square_modified.msh", mesh);
\end{cpp}
which works exactly like the \textit{read} method.

\akantu supports also meshes generated by
DIANA (\url{http://tnodiana.com}), but only in reading mode. A similar
procedure applies where the only
difference is that the \code{MeshIODiana} object should be used
instead of the \code{MeshIOMSH} one. Additional mesh readers can be
introduced in \akantu by coding new \code{MeshIO} objects.

\section{Using \code{Vectors}}

Data in \akantu can be stored in data containers implemented by
the \code{Vector} object. In its most basic usage the \code{Vector} class
implemented in \akantu is similar to the \code{vector} class of
the Standard Template Library (STL) for C++. A simple \code{Vector}
containing a sequence of \texttt{nb\_element} values can be generated with:
\begin{cpp}
  Vector<type> example_vector(nb_element);
\end{cpp}
where \texttt{type} usually is \texttt{Real}, \texttt{UInt} or
\texttt{bool}. Each value is associated to an index, so that data can be
accessed by typing:

\begin{cpp}
  type & val = example_vector(index)
\end{cpp}

\code{Vectors} can also contain a sequence of values for each
index. In that case the number of components of each sequence must be
specified at the \code{Vector} creation.  For example if we want to
create a \code{Vector} to store the coordinates (sequences of three
values) of ten nodes, the appropriate code is the following:
\begin{cpp}
  UInt nb_nodes = 10;
  UInt spatial_dimension = 3;

  Vector<Real> position(nb_nodes, spatial_dimension);
\end{cpp}
In this case the $x$ position of node number 8 will be given by
\texttt{position(7, 0)} (in C++ numbering is starting from 0 and not
1). If the number of components for the sequences is not specified, a
default value of 1 is used.

It is very common in \akantu to loop over vectors to perform a specific
treatment. This ranges from geometric calculation on nodal quantities
to tensor algebra (in constitutive laws for example).
The \code{Vector} object has the possibility to request iterators
in order to make the writing of loops easier and enhance readability.
For instance a loop over the nodal coordinates can be performed like:
\begin{cpp}
  //accessing the nodal coordinates Vector
  Vector<Real> nodes = mesh.getNodes();

  //creating the iterators
  Vector<Real>::iterator<types::RVector> it  = nodes.begin(spatial_dimension);
  Vector<Real>::iterator<types::RVector> end = nodes.end(spatial_dimension);

  for (; it != end; ++it){
    RVector & coords = (*it);

    //do what you need
    ....

  }
\end{cpp}
In that example, each \code{RVector} is a geometrical vector of size \textit{spatial\_dimension}
and the iteration is conveniently performed by the \code{Vector} iterator.

The \code{Vector} object is intensively used to store tensor values.
In that case while constructing the iterator it should be specified
that the returned object type is a matrix. This is done when calling
the begin function. For instance, assuming that we have a
\code{Vector} storing stresses, we can loop over the stored tensors by:

\begin{cpp}
  //creating the iterators
  Vector<Real>::iterator<types::RMatrix> it  = stresses.begin(spatial_dimension,spatial_dimension);
  Vector<Real>::iterator<types::RMatrix> end = stresses.end(spatial_dimension,spatial_dimension);

  for (; it != end; ++it){
    Matrix<Real> & stress = (*it);

    //do what you need
    ....

  }
\end{cpp}
In that last example the Matrix objects are
$spatial\_dimension \times spatial\_dimension$ matrices.
The light objects \code{Matrix} and \code{RVector} can be used and combined
to do most common linear algebra.

In general a mesh consists of several kinds of elements. Consequently the
amount of data to be stored can differ for each element type. The straightforward
example is the connectivity array, namely the sequences of nodes belonging to
each element. In order to easily manage this kind of data, a
particular data structure called \code{ByElementTypeVector} is available.
It is just a group of \code{Vectors}, each associated to an element
type. The following code can retrieve the \code{ByElementTypeVector}
which stores the connectivity arrays for a mesh:
\begin{cpp}
  ByElementTypeVector<UInt> & connectivities = mesh.getConnectivities();
\end{cpp}
Then the specific vector associated to a given element type can be
obtained by
\begin{cpp}
  Vector<UInt> & connectivity_triangle = connectivities[_triangle_3];
\end{cpp}
where the first order 3-node triangular element was used in the presented
piece of code.
