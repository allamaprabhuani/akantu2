\section{Elements\index{Elements}}

The base for every finite element computation is its mesh and the elements that are used within that mesh. What kind of element types can be used depends on the mesh, but also on the dimensionality of the problem (1D, 2D or 3D). In \akantu several isoparametric Langrangian element types are supported. Each of these types is discussed in more detail below, starting with the 1D-elements all the way to the 3D-elements. For each element type the coordinates of the nodes are given in the isoparametric frame of reference, together with the shape functions (and their derivatives) on these respective nodes. Also all the quadrature points within each element are listed (together with the weight that is applied on these points).

%%%%%%%%%% 1D %%%%%%%%%
\subsection{Isoparametric Elements in 1D\index{Elements!1D}}

In \akantu there are two types of isoparametric elements defined in 1D. These element types are called \code{segment\_2} and \code{segment\_3}.

\subsubsection{Segment 2\index{Elements!1D!Segment 2}}

\begin{Element}{1D}
 1  &  \inelemone{-1}  &  $\half\left(1-\xi\right)$  &  \inelemone{-\half} \\
\elemline
 2  &  \inelemone{1}   &  $\half\left(1+\xi\right)$  &  \inelemone{\half}  \\
\end{Element}

\begin{QuadPoints}
\begin{tabular}{l|c}
Coord. \elemcooroned  &  0  \\
\elemline
Weight  &  2  \\
\end{tabular}
\end{QuadPoints}

\subsubsection{Segment 3\index{Elements!1D!Segment 3}}

\begin{Element}{1D}
 1  &  \inelemone{-1}  &  $\half\xi\left(\xi-1\right)$  &  \inelemone{\xi-\half}   \\
\elemline
 2  &  \inelemone{1}   &  $\half\xi\left(\xi+1\right)$  &  \inelemone{\xi+\half}   \\
\elemline
 3  &  \inelemone{0}   &  $1-\xi^{2}$                    &  \inelemone{-2\xi}       \\
\end{Element}

\begin{QuadPoints}
\begin{tabular}{l|cc}
Coord. \elemcooroned  &  $-1/\sqrt{3}$  &  $1/\sqrt{3}$  \\
\elemline
Weight  &  1  &  1  \\
\end{tabular}
\end{QuadPoints}

%%%%%%%%%% 2D %%%%%%%%%
\subsection{Isoparametric Elements in 2D\index{Elements!2D}}

In \akantu there are four types of isoparametric elements defined in 2D. These element types are called \code{triangle\_3}, \code{triangle\_6}, \code{quadrangle\_4} and \code{quadrangle\_8}.

\subsubsection{Triangle 3\index{Elements!2D!Triangle 3}}

\begin{Element}{2D}
 1  &  \inelemtwo{0}{0}  &  $1-\xi-\eta$  &  \inelemtwo{-1}{-1}  \\
\elemline
 2  &  \inelemtwo{1}{0}  &  $\xi$         &  \inelemtwo{1}{0}    \\
\elemline
 3  &  \inelemtwo{0}{1}  &  $\eta$        &  \inelemtwo{0}{1}    \\
\end{Element}

\begin{QuadPoints}
\begin{tabular}{l|c}
Coord. \elemcoortwod  &  \inquadtwo{\third}{\third}  \\
\elemline
Weight  &  1  \\
\end{tabular}
\end{QuadPoints}

\subsubsection{Triangle 6\index{Elements!2D!Triangle 6}}

\begin{Element}{2D}
 1  &  \inelemtwo{0}{0}          &  $-\left(1-\xi-\eta\right)\left(1-2\left(1-\xi-\eta\right)\right)$ & \inelemtwo{1-4\left(1-\xi-\eta\right)}{1-4\left(1-\xi-\eta\right)} \\
\elemline
 2  &  \inelemtwo{1}{0}          &  $-\xi\left(1-2\xi\right)$                                         & \inelemtwo{4\xi-1}{0}                          \\
\elemline
 3  &  \inelemtwo{0}{1}          &  $-\eta\left(1-2\eta\right)$                                       & \inelemtwo{0}{4\eta-1}                   \\
\elemline
 4  &  \inelemtwo{\half}{0}      &  $4\xi\left(1-\xi-\eta\right)$                                     & \inelemtwo{4\left(1-2\xi-\eta\right)}{-4\xi}                      \\
\elemline
 5  &  \inelemtwo{\half}{\half}  &  $4\xi\eta$                                                        & \inelemtwo{4\eta}{4\xi}                       \\
\elemline
 6  &  \inelemtwo{0}{\half}      &  $4\eta\left(1-\xi-\eta\right)$                                    & \inelemtwo{-4\eta}{4\left(1-\xi-2\eta\right)}  \\
\elemline
\end{Element}

\begin{QuadPoints}
\begin{tabular}{l|ccc}
Coord. \elemcoortwod  &  \inquadtwo{\sixth}{\sixth} & \inquadtwo{\twothird}{\sixth} & \inquadtwo{\sixth}{\twothird} \\
\elemline
Weight & \sixth & \sixth & \sixth \\
\end{tabular}
\end{QuadPoints}

\subsubsection{Quadrangle 4\index{Elements!2D!Quadrangle 4}}

\begin{Element}{2D}
 1  &  \inelemtwo{-1}{-1}  &  $\quart\left(1-\xi\right)\left(1-\eta\right)$  &  \inelemtwo{-\quart\left(1-\eta\right)}{-\quart\left(1-\xi\right)} \\
\elemline
 2  &  \inelemtwo{1}{-1}   &  $\quart\left(1+\xi\right)\left(1-\eta\right)$  &  \inelemtwo{\quart\left(1-\eta\right)}{-\quart\left(1-\xi\right)} \\
\elemline
 3  &  \inelemtwo{1}{1}    &  $\quart\left(1+\xi\right)\left(1+\eta\right)$  &  \inelemtwo{\quart\left(1-\eta\right)}{\quart\left(1-\xi\right)}  \\
\elemline
 4  &  \inelemtwo{-1}{1}   &  $\quart\left(1-\xi\right)\left(1+\eta\right)$  &  \inelemtwo{-\quart\left(1-\eta\right)}{\quart\left(1-\xi\right)}  \\
\end{Element}

\begin{QuadPoints}
\begin{tabular}{l|cccc}
\elemcoortwod  &  \inquadtwo{-\invsqrtthree}{-\invsqrtthree}  &  \inquadtwo{\invsqrtthree}{-\invsqrtthree}  
               &  \inquadtwo{\invsqrtthree}{\invsqrtthree}    &  \inquadtwo{-\invsqrtthree}{\invsqrtthree} \\ 
\elemline
Weight & 1 & 1 & 1 & 1 \\
\end{tabular}
\end{QuadPoints}

\subsubsection{Quadrangle 8\index{Elements!2D!Quadrangle 8}}

\begin{Element}{2D}
 1  &  \inelemtwo{-1}{-1}  &  $\quart\left(1-\xi\right)\left(1-\eta\right)\left(-1-\xi-\eta\right)$ 
                           &  \inelemtwo{\quart\left(1-\eta\right)\left(2\xi+\eta\right)}
                                        {\quart\left(1-\xi\right)\left(\xi+2\eta\right)}  \\
\elemline
 2  &  \inelemtwo{1}{-1}  &  $\quart\left(1+\xi\right)\left(1-\eta\right)\left(-1+\xi-\eta\right)$ 
                          &  \inelemtwo{\quart\left(1-\eta\right)\left(2\xi-\eta\right)}
                                       {-\quart\left(1+\xi\right)\left(\xi-2\eta\right)} \\
\elemline
 3  &  \inelemtwo{1}{1}  &  $\quart\left(1+\xi\right)\left(1+\eta\right)\left(-1+\xi+\eta\right)$ 
                         &  \inelemtwo{\quart\left(1+\eta\right)\left(2\xi+\eta\right)}
                                      {\quart\left(1+\xi\right)\left(\xi+2\eta\right)}  \\
\elemline
 4  &  \inelemtwo{-1}{1}  &  $\quart\left(1-\xi\right)\left(1+\eta\right)\left(-1-\xi+\eta\right)$ 
                          &  \inelemtwo{\quart\left(1+\eta\right)\left(2\xi-\eta\right)}
                                       {-\quart\left(1-\xi\right)\left(\xi-2\eta\right)} \\
\elemline
 5  &  \inelemtwo{0}{-1}  &  $\half\left(1-\xi^{2}\right)\left(1-\eta\right)$ 
                          &  \inelemtwo{-\xi\left(1-\eta\right)}
                                       {-\half\left(1-\xi^{2}\right)}  \\
\elemline
 6  &  \inelemtwo{1}{0}  &  $\half\left(1+\xi\right)\left(1-\eta^{2}\right)$
                         &  \inelemtwo{\half\left(1-\eta^{2}\right)}
                                      {-\eta\left(1+\xi\right)}  \\
\elemline
 7  &  \inelemtwo{0}{1}  &  $\half\left(1-\xi^{2}\right)\left(1+\eta\right)$
                         &  \inelemtwo{-\xi\left(1+\eta\right)}
                                      {\half\left(1-\xi^{2}\right)}  \\
\elemline
 8  & \inelemtwo{-1}{0}  &  $\half\left(1-\xi\right)\left(1-\eta^{2}\right)$
                         &  \inelemtwo{-\half\left(1-\eta^{2}\right)}
                                      {-\eta\left(1-\xi\right)}  \\
\end{Element}

\begin{QuadPoints}
\begin{tabular}{l|ccccc}
Coord. \elemcoortwod & \inquadtwo{0}{0} & \inquadtwo{\sqrt{\tfrac{3}{5}}}{\sqrt{\tfrac{3}{5}}} & \inquadtwo{-\sqrt{\tfrac{3}{5}}}{\sqrt{\tfrac{3}{5}}} 
                     & \inquadtwo{-\sqrt{\tfrac{3}{5}}}{-\sqrt{\tfrac{3}{5}}} & \inquadtwo{\sqrt{\tfrac{3}{5}}}{-\sqrt{\tfrac{3}{5}}} \\
\elemline
Weight & 64/81 & 25/81 & 25/81 & 25/81 & 25/81 \\
\elemline
Coord. \elemcoortwod & \inquadtwo{0}{\sqrt{\tfrac{3}{5}}} & \inquadtwo{-\sqrt{\tfrac{3}{5}}}{0} 
                     & \inquadtwo{0}{-\sqrt{\tfrac{3}{5}}} & \inquadtwo{\sqrt{\tfrac{3}{5}}}{0} & \\
\elemline
Weight & 40/81 & 40/81 & 40/81 & 40/81 & \\
\end{tabular}
\end{QuadPoints}

%%%%%%%%%% 3D %%%%%%%%%
\subsection{Isoparametric Elements in 3D\index{Elements!3D}}

\begin{figure}
\begin{center}
\begin{tabular}{m{0.3\textwidth}m{0.3\textwidth}m{0.3\textwidth}}
\subfloat[Tetrahedron-4]{
  \includegraphics[width=0.3\textwidth]{figures/elements/tetrahedron_4}
  \label{fig:elements:tetrahedron4}
} &
\subfloat[Tetrahedron-10]{
  \includegraphics[width=0.3\textwidth]{figures/elements/tetrahedron_10}
  \label{fig:elements:tetrahedron10}
} &
\subfloat[Hexahedron-8]{
  \includegraphics[width=0.3\textwidth]{figures/elements/hexahedron_8}
  \label{fig:elements:hexahedron8}
} 
\end{tabular}
\caption{Schematic depiction of the three 3D isoparametric elements available in \akantu. In each element the node numbering used in \akantu is indicated and also the quadrature points are highlighted (gray spheres)}
\end{center}
\end{figure}


\subsubsection{Tetrahedron 4\index{Elements!3D!Tetrahedron 4}}

\begin{Element}{3D}
 1 & \inelemthree{0}{0}{0} & $1-\xi-\eta-\zeta$ & \inelemthree{-1}{-1}{-1} \\
\elemline 
 2 & \inelemthree{1}{0}{0} & $\xi$              & \inelemthree{1}{0}{0} \\
\elemline 
 3 & \inelemthree{0}{1}{0} & $\eta$             & \inelemthree{0}{1}{0} \\
\elemline 
 4 & \inelemthree{0}{0}{1} & $\zeta$            & \inelemthree{0}{0}{1} \\
\end{Element}

\begin{QuadPoints}
\begin{tabular}{l|c}
Coord. \elemcoorthreed & \inquadthree{\quart}{\quart}{\quart} \\
\elemline
Weight & \sixth \\
\end{tabular}
\end{QuadPoints}

\subsubsection{Tetrahedron 10\index{Elements!3D!Tetrahedron 10}}

\begin{Element}{3D}
  1 & \inelemthree{0}{0}{0} & $\left(1-\xi-\eta-\zeta\right)\left(1-2\xi-2\eta-2\zeta\right)$ 
                            & \inelemthree{4\xi+4\eta+4\zeta-3}{4\xi+4\eta+4\zeta-3}{4\xi+4\eta+4\zeta-3} \\
\elemline
  2 & \inelemthree{1}{0}{0} & $\xi\left(2\xi-1\right)$
                            & \inelemthree{4\xi-1}{0}{0} \\
\elemline
  3 & \inelemthree{0}{1}{0} & $\eta\left(2\eta-1\right)$
                            & \inelemthree{0}{4\eta-1}{0} \\
\elemline
  4 & \inelemthree{0}{0}{1} & $\zeta\left(2\zeta-1\right)$
                            & \inelemthree{0}{0}{4\zeta-1} \\
\elemline
  5 & \inelemthree{\half}{0}{0} & $4\xi\left(1-\xi-\eta-\zeta\right)$
                                & \inelemthree{4-8\xi-4\eta-4\zeta}{-4\xi}{-4\xi} \\
\elemline
  6 & \inelemthree{\half}{\half}{0} & $4\xi\eta$
                                    & \inelemthree{4\eta}{4\xi}{0} \\
\elemline
  7 & \inelemthree{0}{\half}{0} & $4\eta\left(1-\xi-\eta-\zeta\right)$
                                & \inelemthree{-4\eta}{4-4\xi-8\eta-4\zeta}{-4\eta} \\
\elemline
  8 & \inelemthree{0}{0}{\half} & $4\zeta\left(1-\xi-\eta-\zeta\right)$
                                & \inelemthree{-4\zeta}{-4\zeta}{4-4\xi-4\eta-8\zeta} \\
\elemline
  9 & \inelemthree{\half}{0}{\half} & $4\xi\zeta$
                                    & \inelemthree{4\zeta}{0}{4\xi} \\
\elemline
 10 & \inelemthree{0}{\half}{\half} & $4\eta\zeta$
                                    & \inelemthree{0}{4\zeta}{4\eta} \\
\end{Element}

\begin{QuadPoints}
%\newcommand\quada{\tfrac{1}{20}\left(5-\sqrt{5}\right)}
%\newcommand\quadb{\tfrac{1}{20}\left(5+3\sqrt{5}\right)}
\newcommand\quada{\tfrac{\left(5-\sqrt{5}\right)}{20}}
\newcommand\quadb{\tfrac{\left(5+3\sqrt{5}\right)}{20}}
\begin{tabular}{l|cc}
Coord. \elemcoorthreed & \inquadthree{\quada}{\quada}{\quada} & \inquadthree{\quadb}{\quada}{\quada} \\
\elemline
Weight & 1/24 & 1/24 \\
\elemline
Coord. \elemcoorthreed & \inquadthree{\quada}{\quadb}{\quada} & \inquadthree{\quada}{\quada}{\quadb} \\
\elemline
Weight & 1/24 & 1/24 \\
\end{tabular}
\end{QuadPoints}

\subsubsection{Hexahedron 8\index{Elements!3D!Hexahedron 8}}

\begin{Element}{3D}
 1 & \inelemthree{-1}{-1}{-1} & $\eighth\left(1-\xi\right)\left(1-\eta\right)\left(1-\zeta\right)$ 
                              & \inelemthree{-\eighth\left(1-\eta\right)\left(1-\zeta\right)}
                                            {-\eighth\left(1-\xi\right)\left(1-\zeta\right)}
                                            {-\eighth\left(1-\xi\right)\left(1-\eta\right)} \\
\elemline
 2 & \inelemthree{1}{-1}{-1}  & $\eighth\left(1+\xi\right)\left(1-\eta\right)\left(1-\zeta\right)$ 
                              & \inelemthree{ \eighth\left(1-\eta\right)\left(1-\zeta\right)}
                                            {-\eighth\left(1+\xi\right)\left(1-\zeta\right)}
                                            {-\eighth\left(1+\xi\right)\left(1-\eta\right)} \\
\elemline
 3 & \inelemthree{1}{1}{-1}   & $\eighth\left(1+\xi\right)\left(1+\eta\right)\left(1-\zeta\right)$ 
                              & \inelemthree{ \eighth\left(1+\eta\right)\left(1-\zeta\right)}
                                            { \eighth\left(1+\xi\right)\left(1-\zeta\right)}
                                            {-\eighth\left(1+\xi\right)\left(1+\eta\right)} \\
\elemline
 4 & \inelemthree{-1}{1}{-1}  & $\eighth\left(1-\xi\right)\left(1+\eta\right)\left(1-\zeta\right)$ 
                              & \inelemthree{-\eighth\left(1+\eta\right)\left(1-\zeta\right)}
                                            { \eighth\left(1-\xi\right)\left(1-\zeta\right)}
                                            {-\eighth\left(1-\xi\right)\left(1+\eta\right)} \\
\elemline
 5 & \inelemthree{-1}{-1}{1}  & $\eighth\left(1-\xi\right)\left(1-\eta\right)\left(1+\zeta\right)$ 
                              & \inelemthree{-\eighth\left(1-\eta\right)\left(1+\zeta\right)}
                                            {-\eighth\left(1-\xi\right)\left(1+\zeta\right)}
                                            { \eighth\left(1-\xi\right)\left(1-\eta\right)} \\
\elemline
 6 & \inelemthree{1}{-1}{1}   & $\eighth\left(1+\xi\right)\left(1-\eta\right)\left(1+\zeta\right)$ 
                              & \inelemthree{ \eighth\left(1-\eta\right)\left(1+\zeta\right)}
                                            {-\eighth\left(1+\xi\right)\left(1+\zeta\right)}
                                            { \eighth\left(1+\xi\right)\left(1-\eta\right)} \\
\elemline
 7 & \inelemthree{1}{1}{1}    & $\eighth\left(1+\xi\right)\left(1+\eta\right)\left(1+\zeta\right)$ 
                              & \inelemthree{ \eighth\left(1+\eta\right)\left(1+\zeta\right)}
                                            { \eighth\left(1+\xi\right)\left(1+\zeta\right)}
                                            { \eighth\left(1+\xi\right)\left(1+\eta\right)} \\
\elemline
 8 & \inelemthree{-1}{1}{1}   & $\eighth\left(1-\xi\right)\left(1+\eta\right)\left(1+\zeta\right)$ 
                              & \inelemthree{-\eighth\left(1+\eta\right)\left(1+\zeta\right)}
                                            { \eighth\left(1-\xi\right)\left(1+\zeta\right)}
                                            { \eighth\left(1-\xi\right)\left(1+\eta\right)} \\
\end{Element}

\begin{QuadPoints}
\begin{tabular}{l|cccc}
\elemcoortwod  &  \inquadthree{-\invsqrtthree}{-\invsqrtthree}{-\invsqrtthree}  &  \inquadthree{\invsqrtthree}{-\invsqrtthree}{-\invsqrtthree}  
               &  \inquadthree{\invsqrtthree}{\invsqrtthree}{-\invsqrtthree}    &  \inquadthree{-\invsqrtthree}{\invsqrtthree}{-\invsqrtthree} \\ 
\elemline
Weight & 1 & 1 & 1 & 1 \\
\elemline
\elemcoortwod  &  \inquadthree{-\invsqrtthree}{-\invsqrtthree}{\invsqrtthree}   &  \inquadthree{\invsqrtthree}{-\invsqrtthree}{\invsqrtthree}  
               &  \inquadthree{\invsqrtthree}{\invsqrtthree}{\invsqrtthree}     &  \inquadthree{-\invsqrtthree}{\invsqrtthree}{\invsqrtthree} \\ 
\elemline
Weight & 1 & 1 & 1 & 1 \\
\end{tabular}
\end{QuadPoints}

