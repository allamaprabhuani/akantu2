\chapter{Parallel Computation}

This section explains how to launch a parallel computation.
The strategy adopted by \akantu uses a mesh partitioning 
where elements are mapped to processors. Mesh partitions are
then distributed to available processors by adequate routines
as will be described below.  
The sequence of additional operations to be performed by the user are:

\begin{itemize}
\item Initializing the parallel context
\item Partitioning the mesh
\item Distributing mesh partitions
\end{itemize}

After these steps, the \code{Model}
object will proceed to the inter processes communications automatically
without the user having to explicitly take care of them.
In what follows we demonstrate applicability on a 
\code{SolidMechanics} model.

\section{Initializing the parallel context}

In the flavor of the \textbf{Message Passing Interface} (MPI) 
the user must initialize \akantu by forwarding the arguments passed to the program
by using the function \code{initialize}. At the end of the program the 
\akantu should be finalized by calling the function \code{finalize}.  

The initialize function builds a \code{StaticCommunicator} object 
responsible of handling later the inter processes communications.
Amongst various tasks the user can claim the total number of declared 
processors available for computations as well as the process rank through 
the functions \code{getNbProc} and \code{whoAmI} respectively.

An example of the initializing sequence and basic usage of the 
\code{StaticCommunicator} is:

\begin{cpp}
int main(int argc, char *argv[])
{
  akantu::initialize(argc, argv);

  akantu::StaticCommunicator & comm =
  akantu::StaticCommunicator::getStaticCommunicator();
  akantu::Int psize = comm.getNbProc();
  akantu::Int prank = comm.whoAmI();

  ... 

  akantu::finalize();
}
\end{cpp} 

\section{Partitioning the mesh}

After a correct initialization of the processes playing a role in the 
computation the mesh shall be partitioned. We assume that a \code{Mesh} object 
is constructed as presented in section \ref{sect:common:mesh}.
Then a \code{MeshPartition} object must be computed which can be achieved 
by using an appropriate mesh partitioner. At present time the only partitioner 
available is \code{MeshPartitionScotch} which implements the function
\code{partitionate} using the \textbf{Scotch} program 
(\url{http://www.labri.fr/perso/pelegrin/scotch/}). 
This is achieved by the following code

\begin{cpp}
  akantu::Mesh mesh(spatial_dimension);
  akantu::MeshPartition * partition = NULL;
  
  if(prank == 0) {
    akantu::MeshIOMSH mesh_io;
    mesh_io.read("my_mesh.msh", mesh);
    partition = new akantu::MeshPartitionScotch(mesh, spatial_dimension);
    partition->partitionate(psize);
  }
\end{cpp} 

\note{Only the processor of rank $0$ should perform the loading of the mesh file 
  as well as the partitioning operation. Nevertheless the mesh object must by declared 
  for all processors since the mesh distribution will store mesh pieces to that object.}

\section{Distributing mesh partitions}

The distribution of the mesh is done be the \code{SolidMechanicsModel}
automatically through the \code{initParallel} function. 
Thus, after creating a \code{SolidMechanicsModel} with our mesh 
as the initial parameter, the \code{initParallel} method must be called 
receiving the partition as a parameter.

\begin{cpp}
  akantu::SolidMechanicsModel model(mesh);
  model.initParallel(partition);
\end{cpp} 

At that point everything remains as in the sequential case from 
the user point of view. This allows the user to care only 
about his simulation without concern for the parallelism.

An example of explicit dynamic 2d bar in traction in a parallel
context can be found in \code{example/manual/??.cc}.\todo{set the right example file}

\section{Launching a parallel program}

Using \textbf{MPI} a parallel run can be launched from a shell 
using the command 

\begin{cpp}
  mpirun -np #procs program_name parameter1 paramter2 ...
\end{cpp} 
