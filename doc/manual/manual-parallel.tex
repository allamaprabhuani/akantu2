\chapter{Parallel Computation}

This section explains how to launch a parallel computation.
The strategy adopted by \akantu uses a mesh partitioning
where elements are mapped to processors. Mesh partitions are
then distributed to available processors by adequate routines
as will be described below.
The sequence of additional operations to be performed by the user are:

\begin{itemize}
\item Initializing the parallel context
\item Partitioning the mesh
\item Distributing mesh partitions
\end{itemize}

After these steps, the \code{Model}
object proceeds with the interprocess communication automatically
without the user having to explicitly take care of them.
In what follows we show how it works on a \code{SolidMechanics} model.

\section{Initializing the Parallel Context}

The user must initialize \akantu by forwarding the arguments passed to the
program by using the function \code{initialize}, and close \akantu instances
at the end of the program by calling the \code{finalize} function.

\note{This step does not change from the sequential case as it was stated in
Section \ref{sect:common:main}. It only gives a additional motivation in the parallel/MPI context.}

The \code{initialize} function builds a \code{StaticCommunicator} object
responsible for handling the interprocess communications later on.  The
\code{StaticCommunicator} can, for instance, be used to ask the total number of
declared processors available for computations as well as the process rank
through the functions \code{getNbProc} and \code{whoAmI} respectively.

An example of the initializing sequence and basic usage of the
\code{StaticCommunicator} is:

\begin{cpp}
int main(int argc, char *argv[])
{
  initialize("material.dat", argc, argv);

  StaticCommunicator & comm = StaticCommunicator::getStaticCommunicator();
  Int psize = comm.getNbProc();
  Int prank = comm.whoAmI();

  ...

  finalize();
}
\end{cpp}

\section{Partitioning the Mesh}

The mesh is partitioned after the correct initialization of the
processes playing a role in the computation. We assume that a
\code{Mesh} object is constructed as presented in
Section~\ref{sect:common:mesh}.  Then a partition must be computed by
using an appropriate mesh partitioner. At present time, the only
partitioner available is \code{MeshPartitionScotch} which implements
the function \code{partitionate} using the
\textbf{Scotch}\cite{scotch} program.  This is achieved by the
following code

\begin{cpp}
  Mesh mesh(spatial_dimension);
  MeshPartition * partition = NULL;

  if(prank == 0) {
    mesh.read("my_mesh.msh");
    partition = new MeshPartitionScotch(mesh, spatial_dimension);
    partition->partitionate(psize);
  }
\end{cpp}

\note{Only the processor of rank $0$ should load the mesh file to
  partition it. Nevertheless, the \code{Mesh} object must by declared
  for all processors since the mesh distribution will store mesh pieces to that object.}

\section{Distributing Mesh Partitions}

The distribution of the mesh is done automatically by the
\code{SolidMechanicsModel} through the \code{initParallel} method.  Thus,
after creating a \code{SolidMechanicsModel} with our mesh as the initial
parameter, the \code{initParallel} method must be called receiving the partition
as a parameter.

\begin{cpp}
  SolidMechanicsModel model(mesh);
  model.initParallel(partition);
\end{cpp}

After that point, everything remains as in the sequential case from
the user point of view. This allows the user to care only
about his simulation without concern for the parallelism.

An example of an explicit dynamic 2D bar in compression in a parallel
context can be found in \shellcode{\examplesdir/parallel\_2d}.

\section{Launching a Parallel Program}

Using \textbf{MPI} a parallel run can be launched from a shell
using the command

\begin{cpp}
  mpirun -np #procs program_name parameter1 parameter2 ...
\end{cpp}

%%% Local Variables:
%%% mode: latex
%%% TeX-master: "manual"
%%% End:
