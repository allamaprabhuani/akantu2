\section{Heat Transfer model\todo{Guillaume}}

% \subsection{Contact Neighbor Structure}

% The contact neighbor  structure is an interface which is  ment to be heritated
% from in order  to implement different type of  contact neighbor structures. It
% has the following protected attributes:
% \begin{itemize}
% \item id
% \item contact search
% \item master surface
% \item neighbor list
% \item type
% \end{itemize}
% The \emph{id} and the \emph{type}  are characteristics of the contact neighcor
% structure object which  define its id and its  type. The \emph{contact search}
% attribute  is  the associated  contact  search  object  to the  given  contact
% neighbor structure  object. The \emph{master  surface} attribute is the  id of
% the  associated master  surface for  which the  neighbor structure  has  to be
% built. Finally, the neighbor list  is the constructed neighbor structure which
% defines the  impactor nodes  that are  in the neighborhood  of either  a given
% master  node or  a given  master  surface element,  depending on  the type  of
% contact neighbor structure.

% The contact neighbor structure provides the accessor \emph{getNeighborList} to
% the constructed  neighbor list and forces  the heritated classes  to provide a
% public \emph{initNeighborStructure}  function, which initializes  the neighbor
% structure,  as well  as a  public  \emph{update} function,  which updates  the
% neighbor structure.

% \subsubsection{Regular Grid Neighbor Structure}

% The regular  grid neighbor structure builds  a regular grid  around the master
% surface and  uses it in  order to construct  the neighbor list.  This neighbor
% structure can construct both types of neighbor list, the


% \subsection{Implementation of a new solid mechanics problem}

% Let us imagine you want to implement a new material called
% "toto" in akantu. The have to complete the following steps (in
% any order) :
% \begin{enumerate}
% \item
%   Declare a new material in the file
%   \textit{Akantu/model/solid\_mechanics/solid\_mechanics\_model\_material.cc}.
%   You have to had this line after the list of possible cases
% \begin{verbatim}else if(mat\_type == "toto") material =
% parser.readMaterialObject<MaterialToto>(*this,mat_id);
% \end{verbatim}


% \item
%   Include the new material in \textit{Akantu/model/solid\_mechanics/material.hh} \\
%   add the line :
% \begin{verbatim}
% #include "materials/material\_toto.hh"
% \end{verbatim}

% \item
%   For compilation include the new file to compile in
%   \textit{Akantu/CMakelist.txt} by adding
% \begin{verbatim}
% model/solid_mechanics/materials/material_toto.cc
% \end{verbatim}
% \item
%   In \textit{Akantu/model/solid\_mechanics/materials}, create (or copy from
%   an allready existing material) the three following files :\\
%   - material\_toto.cc\\
%   - material\_toto.hh\\
%   - material\_toto\_inline\_impl.cc

% \item
%   Modify the files !
% \end{enumerate}
