\chapter{Introduction}

\akantu means a \textit{little element} in Kinyarwanda, a
Bantu language. From now on it is also an open source
object-oriented \textit{Finite Element} library which aims at being generic and efficient.
\akantu is developed within the LSMS (Computational Solid Mechanics Laboratory, \url{lsms.
epfl.ch}), where research is conducted at the interface of mechanics, material
science, and scientific computing.
The open-source philosophy is important for any
scientific software project. The collaboration
permitted by shared codes enforces sanity when users (and not
only developers) can criticize the implementation details.
\akantu was born with the vision to associate generic programming, robustness
and efficiency while benefiting the open-source visibility.\\

Generic programming paradigm is necessary to allow the easy exploration of mathematical
formulations through algorithmic ideas. Robustness and reliability are naturally
expected from any simulation software, and even more in the context of parallel
computations.  In order to achieve these goals, we made noticeable choices in
the architecture of \akantu. First, we decided to use the object-oriented
paradigm through C++. Then, in order to prevent the extra cost associated with
virtual function calls, we designed the library as a hybrid architecture with
objects at high level layers and vectorization for low level layers. Thus,
Akantu benefits from inheritance and polymorphism mechanisms without the
counterpart of having virtual calls within critical loops.  This coding
philosophy, which was demonstrated in the past to be really
efficient, is quite innovative in the field of \textit{Finite Element} software. \\

This document is appropriate for people willing to use \akantu in order to
perform finite element calculations for solid mechanics, structural mechanics,
contact mechanics or heat transfer. The solid mechanics solver, that is the most
complete and functional part of \akantu, is presented in details throughout this
document in a step by step approach.
%If further help should be required
%requests can be addressed to \href{mailto:akantu@akantu.ch}{akantu@akantu.ch}.
