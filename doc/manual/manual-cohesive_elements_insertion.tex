For cohesive material, \akantu has a pre-defined material selector to assign 
the first cohesive material by default to the cohesive elements which is called 
\code{DefaultMaterialCohesiveSelector} and it inherits its properties from 
\code{DefaultMaterialSelector}. Multiple cohesive materials can be assigned 
using mesh data information (for more details, see \ref{intrinsic_insertion}).

\subsection{Insertion of Cohesive Elements}
<<<<<<< Updated upstream
Cohesive elements are currently compatible only with static simulation
and dynamic simulation with an explicit time integration scheme (see
section~\ref{ssect:smm:expl-time-integr}). They do not have to be inserted when the mesh is generated (intrinsic) but can be added during the simulation (extrinsic). At any time during the simulation, it is possible to access the following energies with the relative function:
\begin{cpp}
  Real Ed = model.getEnergy("dissipated");
  Real Er = model.getEnergy("reversible");
  Real Ec = model.getEnergy("contact");
\end{cpp}

A new model have to be call in a very similar way that the solid mechanics model:
\begin{cpp}
  SolidMechanicsModelCohesive model(mesh);
  model.initFull(SolidMechanicsModelCohesiveOptions(_explicit_lumped_mass, true));
\end{cpp} 


\subsubsection{Extrinsic approach}
The dynamic insertion of extrinsic cohesive elements should be initialized 
in the following way:
\begin{cpp}
  model.updateAutomaticInsertion();
\end{cpp} 
During the simulation, stress has to be checked along each facet in order to insert cohesive elements where thestress criterion is reached.
This check is performed by calling the method \code{checkCohesiveStress}, as 
example before each step resolution:
\begin{cpp}
  model.checkCohesiveStress();
  model.solveStep();
\end{cpp}
The area where stresses are checked and cohesive elements inserted can be limited 
using the method \code{limitInsertion} during initialization. As example, to 
limit insertion in the range $[-1.5, 1.5]$ in the $x$ direction: 
\begin{cpp}
  model.limitInsertion(_x, -1.5, 1.5);
  model.updateAutomaticInsertion();
\end{cpp} 
Additional restrictions with respect to $y$ and $z$ directions can be added as well.

\subsubsection{Intrinsic approach \label{intrinsic_insertion}}
Intrinsic cohesive elements are inserted in the mesh with the method 
\code{insertIntrinsicElements}. Similarly, the range of insertion can me limited 
with \code{limitInsertion}. As example with a static simulation,
\begin{cpp}
  model.limitInsertion(_x, -1.5, 1.5);
  model.insertIntrinsicElements();
\end{cpp} 
Mesh data information becomes vital to the insertion of cohesive elements along 
surface with more sophisticated geometry or when multiple cohesive materials are 
wanted. To do so, cohesive elements can be inserted along a 
specific group of surface elements identified in a GMSH geometry file. This can 
be achieved with the material selector (see section~\ref{sect:smm:materialselector}), in the input file specify the name of these physical groups in the corresponding cohesive materials, and call these material in the \textit{mesh parameters} section. As example, with two physical surfaces named 
\textit{weak\_interface} and \textit{strong\_interface} defined in the GMSH 
geometry file:
\begin{cpp}
...
  material %\emph{cohesive constitutive\_law}% [
     name = weak_interface
     sigma_c = $value$
     ...
  ]
  material %\emph{cohesive constitutive\_law}% [
     name = strong_interface
     sigma_c = $value$
     ...
  ]
  mesh parameters [
     	cohesive_surfaces = weak_interface,strong_interface
  ]
\end{cpp}

In this case, there is no need to call \code{insertIntrinsicElements} anymore 
since the insertion of cohesive elements along physical surfaces is performed 
automatically during \code{initFull} call.    
