\subsection{Insertion of Cohesive Elements}
Cohesive elements are currently compatible only with static simulation and dynamic simualtion with an explicit time integration scheme
(see section~\ref{ssect:smm:expl-time-integr}). They do not have to be
inserted when the mesh is generated but during the simulation. At any time during the simulation, it is possible to access the following energies with the relative function:
\begin{cpp}
  Real Ed = model.getEnergy("dissipated");
  Real Er = model.getEnergy("reversible");
  Real Ec = model.getEnergy("contact");
\end{cpp}

\subsubsection{Extrinsic approach}
The dynamic insertion of extrinsic cohesive elements should be initialized 
the following way:
\begin{cpp}
  SolidMechanicsModelCohesive model(mesh);
  model.initFull(SolidMechanicsModelCohesiveOptions(_explicit_lumped_mass, true));
  model.updateAutomaticInsertion();
\end{cpp} 
During the simulation, stress has to be checked along each facet in order to 
eventually insert cohesive elements.
This check is performed by calling the method \code{checkCohesiveStress}, as 
example before each step resolution:
\begin{cpp}
  model.checkCohesiveStress();
  model.solveStep();
\end{cpp}
The area where stress are checked and cohesive elements inserted can be limited 
using the method \code{limitInsertion} during initialization. As example, to 
limit insertion in the range $[-1.5, 1.5]$ in the $x$ direction: 
\begin{cpp}
  SolidMechanicsModelCohesive model(mesh);
  model.initFull(SolidMechanicsModelCohesiveOptions(_explicit_lumped_mass, true));
  model.limitInsertion(_x, -1.5, 1.5);
  model.updateAutomaticInsertion();
\end{cpp} 
Additional restrictions with respect to $y$ and $z$ directions can be added as well.

\subsubsection{Intrinsic approach \label{intrinsic_insertion}}
Intrinsic cohesive elements are inserted in the mesh with the method 
\code{insertIntrinsicElements}. Similarly, the range of insertion can me limited 
with \code{limitInsertion}. As example with a static simulation,
\begin{cpp}
  SolidMechanicsModelCohesive model(mesh);
  model.initFull(SolidMechanicsModelCohesiveOptions(_static));
  model.limitInsertion(_x, -1.5, 1.5);
  model.insertIntrinsicElements();
\end{cpp} 
Mesh data information becomes vital to the insertion of cohesive elements along 
surface with more sophisticated geometry or when multiple cohesive materials are 
wanted. At this purpose, cohesive elements can be easily inserted along a 
specific group of surface elements identified in a GMSH geometry file. This can 
be achieved, in the input file, by specifying the name of these physical groups 
in the \textit{mesh parameters} section as well as their corresponding cohesive 
materials. As example, with two physical surfaces named 
\textit{weak\_interface} and \textit{strong\_interface} defined in the GMSH 
geometry file:
\begin{cpp}
...
  material %\emph{cohesive constitutive\_law}% [
     name = weak_interface
     sigma_c = $value$
     ...
  ]
  material %\emph{cohesive constitutive\_law}% [
     name = strong_interface
     sigma_c = $value$
     ...
  ]
  mesh parameters [
     	cohesive_surfaces = weak_interface,strong_interface
  ]
\end{cpp}

In this case, there is no need to call \code{insertIntrinsicElements} anymore 
since the insertion of cohesive elements along physical surfaces is performed 
automatically during \code{initFull} call.    
