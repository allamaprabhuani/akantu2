
\subsection{Cohesive laws\todo{Marco}}

\subsubsection{Exponential cohesive law}

Ortiz and Pandolfi proposed this cohesive law in 1999 \ref{ortiz1999}.  The
traction-opening equation for this law is as following:

\begin{equation}
  \label{eq:exponential_law}
  t = e \sigma_c \frac{\delta}{\delta_c}e^{-\delta/ \delta_c}
\end{equation}


This equation is plotted in figure (\ref{fig:smm:CL:ECL}).

 \begin{figure}[!htb]
    \begin{center}
      \includegraphics[width=0.6\textwidth,keepaspectratio=true]{figures/cohesive_exponential.pdf}
      \caption{Exponential cohesive law}
      \label{fig:smm:CL:ECL}
    \end{center}
  \end{figure}



\subsubsection{Static cohesive element}

For  the  static  analysis   of  the  structures  containing  cohesive
elements, the stiffness of the cohesive elements should also be added to the
total      stiffness       of      the      structure.      Therefore,
the  function  \code{assembleStiffnessMatrix}  is written  in  the
\code{MaterialCohesive}.

Considering a typical cohesive element as explained in figure ???, the opening
displacement along the mid-surface can be written as:

\begin{equation}
  \label{eq:opening}
  \delta(\xi) = [[\mat{u}]] \mat{N}(\xi) =
\begin{bmatrix}
u_3-u_0 & u_4-u_1 & u_5-u_2\\
v_3-v_0 & v_4-v_1 & v_5-v_2
\end{bmatrix}
\begin{bmatrix}
N_0(\xi) \\ N_1(\xi) \\ N_2(\xi)
\end{bmatrix} =
\mat{N A U}
\end{equation}

 The \mat{U} , \mat{A} and \mat{N} are as following:

\begin{equation}
  \mat{U} = \left [
\begin{array}{c c c c c c c c c c c c}
u_0 & v_0 & u_1 & v_1 & u_2 & v_2 & u_3 & v_3 & u_4 & v_4 & u_5 & v_5
\end{array}\right ]
\end{equation}


\begin{equation}
  \mat{A} = \left [\begin{array}{c c c c c c c c c c c c}
1 & 0 & 0 & 0& 0 & 0 & -1& 0 & 0 &0 &0 &0\\
0 &1& 0&0 &0 &0 &0 & -1& 0& 0 & 0 &0\\
0 &0& 1&0 &0 &0 &0 & 0& -1& 0 & 0 &0\\
0 &0& 0&1 &0 &0 &0 & 0& 0& -1 & 0 &0\\
0 &0& 0&0 &1 &0 &0 & 0& 0& 0 & -1 &0\\
0 &0& 0&0 &0 &1 &0 & 0& 0& 0 & 0 &-1
\end{array} \right ]
\end{equation}


 \begin{equation}
 \mat{N} = \begin{bmatrix}
N_0(\xi) & 0 & N_1(\xi) &0 & N_2(\xi) & 0\\
0 & N_0(\xi)& 0 &N_1(\xi)& 0 & N_2(\xi)
\end{bmatrix}
\end{equation}

The consistent stiffness matrix for the element is obtained as

\begin{equation}
  \label{eq:cohesive_stiffness}
  \mat{K}    =    \delta    \mat{U}^T    \int_{\Gamma_c}    {\mat{P}^t
    \frac{\partial{\mat{t}}} {\partial{\delta}} \mat{P} d
    \Gamma \Delta \mat{U}}
\end{equation}

In which the  tangent matrix is calculated based  on the equation \ref
{eq:tangent_cohesive} after
performing necessary derivation:

\begin{equation}
  \label{eq:tangent_cohesive}
   \frac{\partial{\mat{t}}} {\partial{\delta}} = \hat{\mat{t}} \otimes
   \frac                       {\partial{(t/\delta)}}{\partial{\delta}}
   \frac{\hat{\mat{t}}}{\delta}+ \frac{t}{\delta}  [ \beta^2 \mat{I} +
   (1-\beta^2) (\mat{n} \otimes \mat{n})]
\end{equation}

 In which $\frac{\partial{(t/\delta)}}{\partial{\delta}}$ in the first
 term is

\begin{equation}
 \frac{\partial{(t/ \delta)}}{\partial{\delta}} = \left\{\begin{array} {l l}
-e  \frac{\sigma_c}{\delta_c^2  }e^{-\delta  /  \delta_c} &  \quad  if
\delta \geq \delta_{max}\\
 0 & \quad if \delta < \delta_{max}, \delta_n > 0
\end{array} \right.
\end{equation}

This      tangent      matrix      is     implemented      in      the
\code{computeTangentStiffness} function for
the exponential law.

A full example for the static analysis of the cohesive elements can be
found in \code{\examplesdir/static\_cohesive}.

%%% Local Variables:
%%% mode: latex
%%% TeX-master: "manual"
%%% End:
