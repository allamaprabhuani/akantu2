\chapter{Input/Output}\index{I\/O}

\section{Input file \label{sect:parser}}

The text input file of a simulation should be precised using the method \code{initialize} which will instantiate the static \code{Parser} object of \akantu. This section explains how to manipulate \code{Parser} objects to input data in \akantu.      
\begin{cpp}
int main(int argc, char *argv[]) {
  initialize("input_files.dat", argc, argv);
  ...
\end{cpp}

\subsection{Akantu Parser}

\akantu file parser has a tree organization.
\begin{itemize}
\item \code{Parser}, the root of the tree, can be accessed using
\begin{cpp}
Parser & parser = getStaticParser();
\end{cpp}
\item \code{ParserSection}, branch of the tree, contains map a of sub-sections (\code{SectionType}, \code{ParserSection}) and a \code{ParserSection *} pointing to the parent section. The user section of the input file can directly be accessed by
\begin{cpp}
const ParserSection & usersect = getUserParser();
\end{cpp} 
\item \code{ParserParameter}, the leaf of the tree, carries data of the input file which can be casted to the correct type with
\begin{cpp}
Real mass = usersect.getParameter("mass");
\end{cpp}
or used directly within an expression
\begin{cpp}
Real dead_weight = 9.81 * usersect.getParameterValue<Real>("mass");
\end{cpp}
\end{itemize}

\subsection{Grammar}

The structure of text input files consists of different sections containing a list of parameters. As example, the file parsed in the previous section will look like
\begin{cpp}
  user parameters [
     mass = 10.5
  ]
\end{cpp}  
Basically every standard arithmetic operations can be used inside of input files as well as the constant \code{pi} and \code{e} and the exponent operator \code{\^{}}. Operations between \code{ParserParameter} are also possible with the convention that only parameters of the current and the parent sections are available. \code{Vector} and \code{Matrix} can also be read according to the \code{NumPy}\cite{numpy} writing convention (a.e. cauchy$\_$stress$\_$tensor = [[$\sigma_{xx}$, $\sigma_{xy}$],[$\sigma_{yx}$,$\sigma_{yy}$]]). 
An example illustrating how to parse the following input file can be found in \code{example$\backslash$io$\backslash$parser$\backslash$example$\_$parser.cc}.
\begin{cpp}
user parameters [
   spatial$\_$dimension = 2
   mesh$\_$file = swiss$\_$cheese.msh
   inner$\_$holes = holes
   outter$\_$crust = crust
   lactostatic$\_$p = 30e3
   stress = [[lactostatic$\_$p,0],[0,lactostatic$\_$p]]
   max$\_$nb$\_$iterations = 100
   precision = 1e-9
]
\end{cpp}
\subsection{Material section \label{sect:io:material}}
The input file should also be used to specify material characteristics (constitutive behavior and material properties). The dedicated material section is then read by \code{initFull} method of \code{SolidMechanicsModel} which initializes the different materials specified with the following convention:
\begin{cpp}
  material %\emph{constitutive\_law}% %\emph{<optional flavor>}% [
     name = $value$
     rho = $value$
     ...
  ]
\end{cpp}
\index{Constitutive\_laws} where \emph{constitutive\_law} is the adopted
constitutive law, followed by the material properties listed one by line in the
bracket (\eg \code{name} and density \code{rho}). Some constitutive laws can
also have an \emph{optional flavor}. More information can be found in sections relative to material
constitutive laws \ref{sect:smm:CL} or in Appendix \ref{app:material-parameters}.

\section{Output data}

\subsection{Generic data}
In this chapter, we address ways to get the internal data in human-readable formats.
The models in \akantu handle data associated to the
mesh, but this data can be split into several \code{Arrays}. For example, the
data stored per element type in a \code{ElementTypeMapArray} is composed of as
many \code{Array}s as types in the mesh.

In order to get this data in a visualization software, the models contain a
object to dump \code{VTK} files. These files can be visualized in software such
as \code{ParaView}\cite{paraview}, \code{ViSit}\cite{visit} or \code{Mayavi}\cite{mayavi}.

The internal dumper of the model can be configured to specify which data fields
are to be written. This is done with the
\code{addDumpField}\index{I\/O!addDumpField} method. By default all the files
are generated in a folder called \code{paraview/}

\begin{cpp}
  model.setBaseName("output"); // prefix for all generated files

  model.addDumpField("displacement");
  model.addDumpField("stress");
  ...

  model.dump()
\end{cpp}

The fields are dumped with the number of components of the memory. For example, in 2D, the memory has 
\code{Vector}s of 2 components, or the $2^{nd}$ order tensors with $2\times2$ components.  
This memory can be dealt with \code{addDumpFieldVector}\index{I\/O!addDumpFieldVector} which always dumps
\code{Vector}s with 3 components or \code{addDumpFieldTensor}\index{I\/O!addDumpFieldTensor} which dumps $2^{nd}$
order tensors with $3\times3$ components respectively. The routines \code{addDumpFieldVector}\index{I\/O!addDumpFieldVector} and
\code{addDumpFieldTensor}\index{I\/O!addDumpFieldTensor} were introduced because of \code{ParaView} which mostly manipulate 3D data.

Those fields which are stored by quadrature point are modified to be seen in the
\code{VTK} file as elemental data. To do this, the default is to average the
values of all the quadrature points.

The list of fields depends on the models (for
\code{SolidMechanicsModel} see table~\ref{tab:io:smm_field_list}).

\begin{table}
  \centering
  \begin{tabular}{llll}
    \toprule
    key          &    type      & support \\
    \midrule
    displacement & Vector<Real> & nodes  \\
    mass         & Vector<Real> & nodes  \\
    velocity     & Vector<Real> & nodes  \\
    acceleration & Vector<Real> & nodes  \\
    force	       & Vector<Real> & nodes  \\
    residual     & Vector<Real> & nodes  \\
    increment     & Vector<Real> & nodes  \\
    {blocked\_dofs}  & Vector<bool> & nodes  \\    
    partitions   & Real         & elements \\
    material\_index & variable  & elements\\    
    strain & Matrix<Real> & quadrature points  \\
    Green strain & Matrix<Real> & quadrature points  \\
    principal strain & Vector<Real> & quadrature points  \\
    principal Green strain & Vector<Real> & quadrature points  \\
    grad\_u & Matrix<Real> & quadrature points  \\    
    stress & Matrix<Real> & quadrature points  \\
    Von Mises stress & Real & quadrature points  \\        
    material\_index & variable  & quadrature points \\
    \bottomrule
  \end{tabular}
  \caption{List of dumpable fields for \code{SolidMechanicsModel}.}
  \label{tab:io:smm_field_list}
\end{table}

\subsection{Cohesive elements' data}
Cohesive elements and their relative data can be easily dumped thanks
to a specific dumper contained in
\code{SolidMechanicsModelCohesive}. In order to use it, one has just
to add the string \code{"cohesive elements"} when calling each method
already illustrated. Here is an example on how to dump displacement
and damage:
\begin{cpp}
  model.setBaseNameToDumper("cohesive elements", "cohesive_elements_output");
  model.addDumpFieldVectorToDumper("cohesive elements", "displacement");
  model.addDumpFieldToDumper("cohesive elements", "damage");
  ...

  model.dump("cohesive elements");
\end{cpp}

\subsubsection{Fragmentation data}

Whenever the \code{SolidMechanicsModelCohesive} is used, it is
possible to dump additional data about the fragments that get formed
in the simulation both in serial and parallel. This task is carried
out by the \code{FragmentManager} class, that takes care of computing
the following quantities for each fragment:
\begin{itemize}
\item index;
\item mass;
\item moments of inertia;
\item velocity;
\item number of elements.
\end{itemize}
These computations can be realized at once by calling the function
\code{computeAllData}, or individually by calling the other public
functions of the class. The data can be dumped to be visualized in
Paraview, or can be accessed within the simulation. An example of
usage is:
\begin{cpp}
  FragmentManager fragment_manager(model);
  fragment_manager.buildAllData();
  ...

  model.addDumpField("fragments");       // this field contains the indices
  model.addDumpField("fragments mass");
  model.addDumpField("moments of inertia");
  model.addDumpField("elements per fragment");
  ...

  for (UInt step = 1; step <= total_steps; ++step) {
    ...

    fragment_manager.buildAllData();
    model.dump();
  }
  ...

  const Array<Real> & fragment_velocities = fragment_manager.getVelocity();
  ...
\end{cpp}
At the end of this example the velocities of the fragments are
accessed with a reference to a \code{const Array<Real>}. The size of
this array is the number of fragments, and its number of components is
the spatial dimension in this case.


\subsection{Advanced dumping}

\subsubsection{Arbitrary fields}
In addition to the predetermined fields from the models and materials, the user
can add any data to a dumper as long as the support is the same. That is to say
data that have the size of the full mesh on if the dumper is dumping the mesh,
or of the size of an element group if it is a filtered dumper.

For this the easiest is to use the ``external'' fields register functions\index{I\/O!addDumpFieldExternal}

The simple case force nodal and elemental data are to pass directly the data
container itself if it as the good size.
\begin{itemize}
\item For nodal fields :
\begin{cpp}
  template<class T>
  addDumpFieldExternal(const std::string & field_name,
                       const Array<T> & field);
\end{cpp}

It is assumed that the array as the same size as the number of nodes in the mesh

\begin{cpp}
  Array<T> nodal_data(nb_nodes, nb_component);
  mesh.addDumpFieldExternal("my_field", nodal_data);
\end{cpp}

\item For elemental fields :

\begin{cpp}
  template<class T>
  addDumpFieldExternal(const std::string & field_name,
                       const ElementTypeMapArray<T> & field);
\end{cpp}

It is assumed that the arrays in the map have the same sizes as the element numbers in the mesh for element types of dimension \code{spatial\_dimension}.

\begin{cpp}
  ElementTypeMapArray<Real> elem_data;
  elem_data.initialize(mesh,
                       _nb_component = 1,
                       _with_nb_element = true);
  mesh.addDumpFieldExternal("my_field", elem_data);
\end{cpp}
\end{itemize}

If some changes have to be applied on the data as for example a padding for
\code{ParaView} vectors, this can be done by using the
field interface.

\begin{cpp}
  mesh.addDumpFieldExternal(const std::string & field_name,
                             std::shared<dumpers::Field> & field);
\end{cpp}

All these functions use the default dumper registered in the mesh but  also have the \code{ToDumper} variation with the dumper name specified.
For example:
\begin{cpp}
  mesh.addDumpFieldExternalToDumper("cohesive elements",
                                    "my_field",
                                    my_field);
\end{cpp}


An example of code presenting this interface is present in the
\shellcode{\examplesdir/io/dumper}. This interface is part of the
\code{Dumpable} class from which the \code{Mesh} inherits.

\subsubsection{Creating a new dumper}

You can also create you own dumpers, \akantu uses a third-party library in order
to write the output files, \code{IOHelper}. \akantu supports the \code{ParaView}
format and a Text format defined by \code{IOHelper}.

This two files format are handled by the classes
\code{DumperParaview}\index{I\/O!DumperParaview} and
\code{DumperText}\index{I\/O!DumperText}.

In order to use them you can instantiate on of this object in your code. This
dumper have a simple interface. You can register a mesh
\code{registerMesh}\index{I\/O!registerMesh},
\code{registerFilteredMesh}\index{I\/O!registerFilteredMesh} or a field,
\code{registerField}\index{I\/O!registerField}.

An example of code presenting this low level interface is present in the
\shellcode{\examplesdir/io/dumper}. The different types of \code{Field} that can
be created are present in the source folder \shellcode{src/io/dumper}.

%%% Local Variables:
%%% mode: latex
%%% TeX-master: "manual"
%%% End: 
