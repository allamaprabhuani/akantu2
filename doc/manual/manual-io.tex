
\chapter{Input/Output}\index{I\/O}

\section{Generic data}
In this chapter, we address ways to get the internal data in human-readable formats.
The models in \akantu handle data associated to the
mesh, but this data can be split into several \code{Arrays}. For example, the
data stored per element type in a \code{ElementTypeMapArray} is composed of as
many \code{Array}s as types in the mesh.

In order to get this data in a visualization software, the models contain a
object to dump \code{VTK} files. These files can be visualized in software such
as \code{ParaView}\cite{paraview}, \code{ViSit}\cite{visit} or \code{Mayavi}\cite{mayavi}.

The internal dumper of the model can be configured to specify which data fields
are to be written. This is done with the
\code{addDumpField}\index{I\/O!addDumpField} method. By default all the files
are generated in a folder called \code{paraview/}

\begin{cpp}
  model.setBaseName("output"); // prefix for all generated files

  model.addDumpField("displacement");
  model.addDumpField("stress");
  ...

  model.dump()
\end{cpp}

The fields are dumped with the number of components of the memory. For example, in 2D, the memory has 
\code{Vector}s of 2 components, or the $2^{nd}$ order tensors with $2\times2$ components.  
This memory can be dealt with \code{addDumpFieldVector}\index{I\/O!addDumpFieldVector} which always dumps
\code{Vector}s with 3 components or \code{addDumpFieldTensor}\index{I\/O!addDumpFieldTensor} which dumps $2^{nd}$
order tensors with $3\times3$ components respectively. The routines \code{addDumpFieldVector}\index{I\/O!addDumpFieldVector} and
\code{addDumpFieldTensor}\index{I\/O!addDumpFieldTensor} were introduced because of Paraview which mostly manipulate 3D 
data.

Those fields which are stored by quadrature point are modified to be seen in the
\code{VTK} file as elemental data. To do this, the default is to average the
values of all the quadrature points.

The list of fields depends on the models (for
\code{SolidMechanicsModel} see table~\ref{tab:io:smm_field_list}).

\begin{table}
  \centering
  \begin{tabular}{llll}
    \toprule
    key          &    type      & support \\
    \midrule
    displacement & Vector<Real> & nodes  \\
    velocity     & Vector<Real> & nodes  \\
    acceleration & Vector<Real> & nodes  \\
    force	       & Vector<Real> & nodes  \\
    residual     & Vector<Real> & nodes  \\
    boundary     & Vector<bool> & nodes  \\
    mass         & Vector<Real> & nodes  \\
    partitions   & Real         & elements \\
    stress & Matrix<Real> & quadrature points  \\
    Von Mises stress & Real & quadrature points  \\
    grad\_u & Matrix<Real> & quadrature points  \\
    strain & Matrix<Real> & quadrature points  \\
    principal strain & Vector<Real> & quadrature points  \\
    Green strain & Matrix<Real> & quadrature points  \\
    principal Green strain & Vector<Real> & quadrature points  \\
    \textit{material internals} & variable  & quadrature points  \\
    \bottomrule
  \end{tabular}
  \caption{List of dumpable fields for \code{SolidMechanicsModel}.}
  \label{tab:io:smm_field_list}
\end{table}

The user can also register external fields which have the same mesh as the mesh from the model as support. To do this, an object of type \code{Field} has to be created.\index{I\/O!addDumpFieldExternal}

\begin{itemize}
\item For nodal fields :
\begin{cpp}
  Vector<T> vect(nb_nodes, nb_component);
  Field field = new DumperIOHelper::NodalField<T>(vect));
  model.addDumpFieldExternal("my_field", field);
\end{cpp}

\item For elemental fields :
\begin{cpp}
  ElementTypeMapArray<T> arr;
  Field field = new DumperIOHelper::ElementalField<T>(arr, spatial_displacement));
  model.addDumpFieldExternal("my_field", field);
\end{cpp}
\end{itemize}

\section{Cohesive elements' data}
Cohesive elements and their relative data can be easily dumped thanks
to a specific dumper contained in
\code{SolidMechanicsModelCohesive}. In order to use it, one has just
to add the string \code{"cohesive elements"} when calling each method
already illustrated. Here is an example on how to dump displacement
and damage:
\begin{cpp}
  model.setBaseNameToDumper("cohesive elements", "cohesive_elements_output");
  model.addDumpFieldVectorToDumper("cohesive elements", "displacement");
  model.addDumpFieldToDumper("cohesive elements", "damage");
  ...

  model.dump("cohesive elements");
\end{cpp}


\section{Advanced dumping}

In addition to the predetermined fields from the models and materials, the user
can add any data to a dumper as long as the support is the same. That is to say
data that have the size of the full mesh on if the dumper is dumping the mesh,
or of the size of an element group if it is a filtered dumper.



%%% Local Variables: 
%%% mode: latex
%%% TeX-master: "manual"
%%% End: 
