\chapter{Input/output}\index{I\/O}

In this chapter, the problem of getting the internal data in a human readable way
will be addressed. The models in \akantu handle data associated to your
mesh, but this data can be split in many \code{Vectors}. For example, the
data stored per element type in a \code{ByElementTypeVector} is composed by as
many vectors as types in the mesh.

To help you get this data into a visualization software, the models contain a
object to dump \code{VTK} files. These files can be visualized in software such
as \code{ParaView}\cite{paraview}, \code{ViSit}\cite{visit} or \code{Mayavi}\cite{mayavi}.

The internal dumper of the model can be configured to specify which data fields
are to be output. This is done with the
\code{addDumpField}\index{I\/O!addDumpField} method. By default all the files
are generated in a folder called \code{paraview/}

\begin{cpp}
  model.setBaseName("output"); // prefix for all generated files

  model.addDumpField("displacement");
  model.addDumpField("stress");
  ...

  model.dump()
\end{cpp}

The fields are dumped with the number of component they have in memory. In 2D, for
example, vectors have 2 components and $2^{nd}$ order tensors have $2\times2$
components.  This can be changed by using
\code{addDumpFieldVector}\index{I\/O!addDumpFieldVector} which always dumps
vectors with 3 components or
\code{addDumpFieldTensor}\index{I\/O!addDumpFieldTensor} which dumps $2^{nd}$
order tensors with $3\times3$ components.

The fields which are stored by quadrature points are modified to be seen in the
\code{VTK} file as elemental data. To do this, the default is to average the
values of the quadrature points.

The list of fields depends on the models.

\paragraph{\code{SolidMechanicsModel}:\index{I\/O!SolidMechanicsModel}}\hfill
\vspace*{0.2cm}

\begin{tabular}{llll}
  \toprule
  key          &    type      & support \\
  \midrule
  displacement & Vector<Real> & nodes  \\
  velocity     & Vector<Real> & nodes  \\
  acceleration & Vector<Real> & nodes  \\
  force	     & Vector<Real> & nodes  \\
  residual     & Vector<Real> & nodes  \\
  boundary     & Vector<bool> & nodes  \\
  mass         & Vector<Real> & nodes  \\
  partitions   & Real         & elements \\
  stress & Matrix<Real> & quadrature points  \\
  strain & Matrix<Real> & quadrature points  \\
  \textit{material internals} & variable  & quadrature points  \\
\bottomrule
\end{tabular}


The user can also register external fields which have the same mesh as the mesh from the model as support. To do this an object of type \code{Field} as to be created.\index{I\/O!addDumpFieldExternal}

\begin{itemize}
\item For nodal fields :
\begin{cpp}
  Vector<T> vect(nb_nodes, nb_component);
  Field field = new DumperIOHelper::NodalField<T>(vect));
  model.addDumpFieldExternal("my_field", field);
\end{cpp}

\item For elemental fields :
\begin{cpp}
  ByElementTypeVector<T> vect;
  Field field = new DumperIOHelper::ElementalField<T>(vect, spatial_displacement));
  model.addDumpFieldExternal("my_field", field);
\end{cpp}
\end{itemize}


%%% Local Variables: 
%%% mode: latex
%%% TeX-master: "manual"
%%% End: 
